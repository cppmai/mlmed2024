\documentclass{article}
\usepackage{graphicx}
% \usepackage{float}
\title{Report on COVID-QU-Ex Dataset}
\author{BI12-263 Chau Phan Phuong Mai}
\begin{document}
\maketitle

\section{Introduction}
\noindent Radiological Technologies is widely used in medical field to assist doctor to see inside in human body. It contains some kind of medical images such as X-ray, Computed Tomography (CT), Mammography, etc. Out of them, X-ray is the most important tools in modern medicine. \\
\noindent The discovery of X rays in 1985 was the beginning of a revolutionary change in our understanding of physical word. Using radiation which has wavelengths ranging from 0.01 to 10 nanometers, X-ray provides images of the bones, organs, soft tissue and vessels that comprise the human body. In the medical domain, the X-ray can assist the doctor to detect to do some tasks: analyse the structures of composite bodies; detect or even destroy the abnormal tissues.  \\
\noindent In this report, I will explore the Lung X-ray images related to COVID-19 and use U-Net model to segmentation lung location.\\

\section{Dataset}

\noindent The COVID-QU-Ex Database from Kaggle was used in this report. It has complied by Qatar University. There are 33 920 chest X-ray (CXR) images and their ground-truth lung segmentation mask which includes: 11 956 COVID\_19, 11 263 Non-COVID infections (Viral or Bacterial Pneumonia) and 10 701 Normal. \\
\begin{figure}
    \centering
    \includegraphics[width=0.6\linewidth]{lung.png}
    \caption{Examples of the dataset.}
    \label{fig:enter-label}
\end{figure}
\noindent
It is clear from the Figure 1. that people with pulmonary disease often have a high opacity, which makes segmentation of lung from chest X-rays more difficult. 

\section{Method}

\noindent the U\_Net structure is able to extract the features and spatial characteristics of images of chest region, which helps the neural network to segment the lung regions from chest images.  \\

\end{document}
